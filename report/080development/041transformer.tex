\section{Transformer}

Clang provides an interface into its C++ AST called LibTooling. LibTooling is aimed at developers who want to build standalone tools and services that run clang tools.\cite{LibToolingClang17}

Tools that use LibTooling run what is called 'FrontendActions' over the specified code. It is through these frontend actions the tool can interact with the source code. The LibTooling tools work by parsing the command line options the tool is invoked with through the LLVM command line parser. These options can be customised to allow the tools to work in a user-defined manner. 

After the command options are parsed, the tool needs an ASTMatcher. The ASTMatcher is the DSL formula for traversing the Clang C++ AST. There are many different matchers which allow for extensive and custom matching of the AST.\cite{ASTMatcherReference,MatchingClangAST} If the matcher finds a valid C++ AST then a call to a user-defined \textit{Consumer} callback is made with the matched AST. This allows the user to further process the AST and to perform the task the tool is meant to solve.
It is also possible to write custom matchers if the built-in ones are not enough to solve the task.

LibTooling is used in many different types of tools such as static analysis tools, code refactoring, language standard migration tools and much more.\cite{ExternalClangExamples}

Because the Clang AST is so comprehensive, the Clang development team has provided a construct called a \cppinline{Transformer}. A Transformer is a way to couple a \textit{rule} together with a consumer callback. A rule is a combination of an AST matcher, a \textit{change} and potentially some metadata. A rule could be specified as follows: ''The name 'MkX' is not allowed in our code base, so find all functions with the name 'MkX' and change it to 'MakeX'''.This could be translated into LibTooling rule:

\begin{listing}[H]
    \begin{cppcode}
auto RenameFunctionWithInvalidName = makeRule(
    functionDecl(hasName("MkX")).bind("fun"), 
    changeTo(
        clang::transformer::name("fun"), 
        cat("MakeX")
    ),
    cat("The name ``MkX`` is not allowed for functions; the function has been renamed")
);
    \end{cppcode}
    \caption{Example of a LibTooling Rule that renames a method 'MkX' to 'MakeX' and provides a reason for the renaming.}
    \label{code:080dev:TransformerRuleExample}
\end{listing}

Where \cppinline{functionDecl(hasName("MkX")).bind("fun")} is the ASTMatcher that matches all function declarations with the name ''MkX'' and binds it to the name ''fun''. The line \\\cppinline{changeTo(clang::transformer::name("fun"), cat("MakeX"))} is specifying the change to be made, which is to change the name of match bound to ''fun'' to ''MakeX''. The last line in \cref{code:080dev:TransformerRuleExample} is the metadata that is associated with this rule. When the rule matches the wanted AST expression it creates a \cppinline{AtomicChange} according to the specified change. 

This is then where the \cppinline{Transformer} comes in. The transformer takes the resulting \cppinline{AtomicChange} and the provided metadata and calls a \textit{Consumer} callback with the findings. In this callback, the developers of the tool are then able to provide diagnostic messages and make source code changes if that is what the tool is providing.\cite{ClangTransformerTutorial}

In the official documentation from Clang, the transformer shown in \cref{code:080dev:TransformerRuleExample} is shown and explained. However, there is no official documentation on how to invoke the transformer and make it write the changes into the existing source code. 

The public repository for this R\&D project contains an \href{https://github.com/mortenhaahr/RD/tree/main/examples}{example folder}. This folder contains multiple examples which show how to create Clang tools that write the results from a \cppinline{Transformer} onto disk.\cite{kristensenMortenhaahrRD2023} In the examples the \cppinline{RefactoringTool} helper class is used to facilitate the source-code changes. This helper class is not documented in Clang's official documentation, but it contains helper methods that make source code changes easier for the developers. 



\section{Example of a C-style to std::array converter}

This section will show an example of a tool that can find all constant-sized arrays and convert them to the C++11 \cppinline{std::array} type. This transformation is recommended in the C++ core guidelines.\cite{CoreGuidelines}

When creating a tool there are multiple steps. The steps are:
\begin{enumerate}
    \item Matching the AST nodes
    \item Extracting information from the nodes
    \item Creating the replacement source code
\end{enumerate}

The \href{https://github.com/mortenhaahr/RD}{github repo for this R\&D project} contains an example project with all the source code described in this section under "examples/c\_style\_conversion\_tool".

\subsection{Matching the AST}

The first step in the process is to find the AST nodes of interest. In this example the declarations of arrays with a constant size. In order to specify the matcher needed it is essential to have some valid C++ code that contains the source code that needs to be changed. For this example tool, the constant array types are specified in the "examples/c\_style\_conversion\_tool/input.cpp" file on github.

In order to get an idea of the type of AST nodes that are interesting to match, the AST for the entire file is "dumped" through clang by running \bashinline{clang -cc1 -ast-dump input.cpp}. This command will show all of the AST for the file. The output of the command can be seen on 


From the output it can be seen that a constant-sized array declaration consists of a DeclStmt that is a VarDecl


\subsection{Getting information out of the nodes}

\subsubsection{Stencils}

\subsection{Creating the replacements}

\subsubsection{Range selection}

\subsection{Combining it all}