\chapter{Conclusion}
This R\&D project focuses on exploring the capabilities of the LibTooling library for automatic source-code generation.
The report began with an introduction to LibTooling and provides detailed instructions for configuring a development environment.
It then proposed a structured approach to tool development using LibTooling, which was followed in the creation of three distinct tools.

The first tool was a simple renaming tool that provided a practical introduction to developing tools with LibTooling.

The second tool focused on C-style array conversion and can be used to refactor existing codebases into more modern and safe code by adopting \cppinline{std::array}.
During its development, several subtle semantic details regarding the use of C-style arrays as function parameters were revealed.
These details highlighted the tool's inability to handle all edge cases of the transformation.
As a result, a more modular approach was proposed for the tool, narrowing its scope while creating opportunities for other tools to optimize the codebase further.

The third tool aimed to automate the generation of a ``to\_string'' function for enum declarations.
Two different approaches were explored during the development process.
The first approach implemented the functionality in a single step, while the second approach followed a modular structure and divided the implementation into two steps.
Ultimately, the multi-step approach was favoured for its modularity and ease of development.
Surprisingly, it also exhibited similar or better performance than the single-step implementation, making it the preferred approach for the development of future tools.


In conclusion, LibTooling proved to be an effective library for developing automatic source-code generation tools.
Leveraging the lexical- and syntax analysis of Clang ensures that the initial C++ specification is valid, providing a solid foundation for tool development.
The fine-grained AST  allows for powerful transformations, but the granularity of the AST can present challenges, as the sheer number of components can be overwhelming to work with.
Nevertheless, with careful consideration and appropriate strategies, the library enables the development of sophisticated tools that can greatly assist in code generation and refactoring tasks.