\section{Installing LLVM and Clang}
One might think installing LLVM and Clang should be a straightforward process but in reality, it can be quite complex. Therefore, this section means to describe the process that was used during the R\&D project. The section is heavily inspired by \cite{TutorialBuildingTools} but more specialized to account for the concrete project.

The process of compiling LLVM and Clang can be considered a two-step process where LLVM, Clang and the associated libraries initially are compiled using an arbitrary C++ compiler and then compiled using Clang itself.

Before compiling the projects, one must install the needed tools, which include ''CMake'' and ''Ninja''. They can be installed using a package manager or by compiling it locally through \url{https://github.com/martine/ninja.git} and \url{git://cmake.org/stage/cmake.git}. Furthermore, one needs to have a working C++ compiler installed. The rest of this section assumes the compiler GCC is installed.


LLVM and Clang can then be compiled for the first time using GCC. Assuming the terminal is used, this can be done as seen in \cref{code:080dev:llvm_compile_init}.
First, the LLVM repository is cloned which also contains the Clang project.
Line 2 - 4 goes inside the repository and sets up the build folder.
Line 5 uses CMake to configure the project where Ninja is used as the generator\footnote{What is a generator}, Clang and Clang Tools are enabled, tests are enabled and it should be built in release mode. The final line instructs Ninja to build the projects. Note that it is possible to compile without ''clang-tools-extra'' but these projects can be helpful during development as they can be used as inspiration for developing tools using LibTooling. Furthermore, it is also possible to skip building and running the tests but it is recommended to ensure that the build process succeeded.

\begin{listing}[H]
    \begin{bashcode}
git clone https://github.com/llvm/llvm-project.git
cd llvm-project
mkdir build
cd build
cmake -G Ninja ../llvm -DLLVM_ENABLE_PROJECTS="clang;clang-tools-extra" -DLLVM_BUILD_TESTS=ON -DCMAKE_BUILD_TYPE=Release
ninja
    \end{bashcode}
    \caption{Bash commands to }
    \label{code:080dev:llvm_compile_init}
\end{listing}

