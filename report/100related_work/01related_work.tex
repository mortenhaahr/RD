\chapter{Related work}

The focus of this project has been on deterministic source-code generation using user-provided source-code as the specification. This is, however, not the only methodology of source-code generation.

Similar work was made in \cite{sadovnikovUsingClangSource} which provides a presentation on how LibTooling can be used as a deterministic source-code generation tool.
However, \cite{sadovnikovUsingClangSource} provides a more superficial overview of this, and focuses on the general challenges associated with developing such tools.
For instance, the work in \cite{sadovnikovUsingClangSource} implements a simpler enum-to-string tool\footnote{
    To compare, their matcher simply consists of \cppinline{enumDecl(isExpansionInMainFile())}.
} where existing ``to\_string'' functions are not taken into consideration, meaning that the tool can not be used consecutively on a code base.

One example of a different approach to source-code generation comes from microcontroller manufacturers such as STMicroelectronics \cite{stmSTMicroelectronicsOurTechnology}.
Tools such as STM32CubeIDE enable developers to conveniently add startup and configuration code to their codebases through a user-friendly checkbox tool \cite{stmSTM32CubeDevelopmentSoftware}.
The configuration code is pre-written by the manufacturers and it is used to specify the state of the peripherals inside the microcontrollers.\\
This form of source-code generation is also deterministic but raises the level of abstraction by defining a predetermined set of options. These options serve as the specification instead of the users' code.
The limited set of options simplifies the number of edge cases that must be considered but at the cost of flexibility.
In contrast, the tools developed during this project are generic allowing them to be used with any C++ specification, as demonstrated with the JSON library \cite{lohmannJSONModern2022} in \cref{chp:090:tests}.

Another approach to automatic programming and source-code generation is to use tools that are based on a probabilistic model.
A probabilistic model can be very useful for source-code generation, even though it is non-deterministic \cite{chenEvaluatingLargeLanguage2021}.
This type of source-code generation is often employed to generate code from natural language (NL)\footnote{
    An example of a NL could be English.
}
descriptions of problems, as deterministic models can be impractical \cite{alonStructuralLanguageModels2020}.
Research on probabilistic automatic programming has been ongoing for years and the newest tools in the field are actively being used by the public \cite{WhatChatGPTWhy, johnmaedaChoosingLLMModel2023}.
This type of model usually consists of some kind of neural network which is used to process the specification. 
The result of the processing can then, in some cases, be used as a specification for a deterministic model as shown in \cite{yinSyntacticNeuralModel2017}.\\
The distinction between deterministic and probabilistic tools lies in their approach to transformation. Deterministic tools strictly adhere to a predefined set of rules and only execute those specific transformations. Consequently, deterministic tools may not be able to handle all types of inputs or prompts.
In contrast, probabilistic tools operate based on probabilities and strive to respond to all inputs, although the quality of the responses may vary.\\
As a result, deterministic tools may be more suitable to deploy in a completely automated environment where precise and reliable transformations are required.
On the contrary, probabilistic models may be more suitable for assisting humans in the development process, where their flexibility and ability to handle a wide range of inputs can be valuable.