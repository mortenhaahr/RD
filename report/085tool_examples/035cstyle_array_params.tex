
\section{C-style array parameter converter} \label{sec:080:035:cstyleArrayParm}

So far the tool was able to rewrite variable declarations of C-style arrays into their corresponding \cppinline{std::array} implementations. However, if the variable that was transformed is passed as an argument to a function, the resulting C++ code is invalid.
As a result, the tool should be extended to change the declaration of such functions to accept \cppinline{std::array}s\footnote{
    A different and potentially better approach is discussed in \cref{subsec:095disc:arrayParm}
}.

\subsection{Command line parsing}

The functionality described in this section is an extension of the C-style array converter that is focusing solely on the parameter conversion, so it has no changes to the command line parsing as compared to the original tool.

\subsection{AST node matching}

For this augmentation of the existing tool, a new matcher has to be created. The matcher must find all the function parameter declarations which are constant arrays. The matcher should additionally still extract the needed information like the variable declaration converter (\cref{subsec:085tool_example:030cstyle:node_matching}).

There is an existing matcher in the LibTooling catalogue that will match a function parameter declaration. That matcher is called \cppinline{parmVarDecl} and it will be used in this matcher instead of the \cppinline{declaratorDecl} used previously.

The built-in matcher \cppinline{parmVarDecl} provides a way to match all parameter declarations, i.e., parameters of functions or methods. For this tool, the parameters of interest are only \cppinline{ConstantArrayType}s. However, writing such a matcher is not trivial, due to certain aspects of the language.\\
At the lowest level of abstraction, a C-style array is essentially a pointer to a block of contiguous memory.
This means that when a C-style array is used in certain contexts, it can be treated as a pointer, which is commonly referred to as array decaying.
Clang AST uses this concept and represents C-style array parameters as pointers instead of arrays.
The challenge was therefore to differentiate between normal pointer types from decayed ones.\\
Fortunately, there is a built-in matcher called \cppinline{decayedType}, which allows matching on decayed pointers.

Matching on all \cppinline{DecayedType}s, however, is too general, as it is also possible for the other array types (\cref{subsec:085tool_example:030cstyle:node_matching}) and function types to decay into pointers.
Unfortunately, a matcher that only matches on \cppinline{DecayedType}s that were previously \cppinline{ConstantArrayType}s does not exist within the built-in libraries. It is, however, possible to pull the information from a node, as a \cppinline{DecayedType} can be cast to an \cppinline{AdjustedType}, which is a type that was implicitly adjusted based on the semantics of the language, i.e., due to array decay in this scenario. \cppinline{AdjustedType} contains meta-information of both the new type and the original type.

Since \cppinline{AdjustedType} exists, it is possible to write a custom matcher that finds the original type of the node, and thereby checks if the node decayed from a \cppinline{ConstantArrayType}.
Creating a custom matcher is done through helper macros in the library. There are many different helper macros, which allow the creator of the matcher to fine-tune the matcher to the exact needs. This flexibility also makes it somewhat complicated for first-time developers, as there are many options to sort through. In this case, the matcher is provided with an \cppinline{AdjustedType} node, which has to be filtered on the original type. The original type has to be compared with another type. This means that a parameter is needed for the matcher. The correct macro for this type of matcher is the \cppinline{AST_MATCHER_P} macro. This macro allows the user to specify the input node type and a single parameter which is given to the matcher. Because this matcher will compare types, the type of the input parameter is a \cppinline{Matcher<QualType>}. The signature of the matcher is shown on \cref{code:085tool_ex:035array_param:match_signiture}.

\begin{listing}[H]
    \begin{cppcode}
AST_MATCHER_P(AdjustedType, hasOriginalType, ast_matchers::internal::Matcher<QualType>, InnerType) {
    ...
}
    \end{cppcode}
    \caption{Signiture of the custom matcher \cppinline{hasOriginalType}.}
    \label{code:085tool_ex:035array_param:match_signiture}
\end{listing}

The implementation of the matcher is quite simple. As mentioned earlier, the matcher needs to extract the original type from the \cppinline{AdjustedType} and compare it to the type provided as the parameter of the matcher. This can be achieved with the code shown on \cref{code:085tool_ex:035array_param:match_impl}.

\begin{listing}[H]
    \begin{cppcode}
return InnerType.matches(Node.getOriginalType(), Finder, Builder);
    \end{cppcode}
    \caption{Implementation of the custom matcher \cppinline{hasOriginalType}.}
    \label{code:085tool_ex:035array_param:match_impl}
\end{listing}

The \cppinline{Finder} and \cppinline{Builder} variables are common across all matchers. The \cppinline{Finder} variable is the \cppinline{MatchFinder} variable created by the tool, and it is responsible for calling the callbacks when a valid match has been found. The \cppinline{Builder} variable is used to bind nodes to specific names through the \cppinline{.bind(NAME)} construct used in the tools.

With all the building blocks in place, the C-style array parameter matcher can be constructed. It looks similar to the variable C-style array matcher and is therefore very expressive. The C-style array parameter matcher can be seen on \cref{code:085tool_ex:035array_param:parm_matcher}.

\begin{listing}[H]
    \begin{cppcode}
auto ParmConstArrays = parmVarDecl(
        isExpansionInMainFile(),
        hasType(
            decayedType(hasOriginalType(constantArrayType().bind("parm")))),
        hasTypeLoc(typeLoc().bind("parmLoc")) 
    ).bind("parmDecl");
    \end{cppcode}
    \caption{C-style array parameter matcher.}
    \label{code:085tool_ex:035array_param:parm_matcher}
\end{listing}

\subsection{Node data processing}

The goal of this tool expansion is the same as it was for C-style array variables, it just has to use \cppinline{ParmVarDecl}s instead of \cppinline{VarDecls}. \cppinline{ParmVarDecl}s are a specialisation of \cppinline{VarDecl}s, so all the node processing is the same for both types. Therefore the node processing can be reused for this tool expansion, and the different match filtering is all that is needed\footnote{The source-code for the converted \cppinline{std::array} parameter should be a reference, which means that a single ``\&'' was added to the generated text}.

\subsection{Handling the results}

As this is an expansion to the C-style array converter tool, the handling of the results is identical.