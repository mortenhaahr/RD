% Discuss the semantic changes that were made through this C-style array tool - focus on the calling convention to methods.

\section{Semantic considerations}
In this section, some of the semantic considerations that were processed and discovered during the development of the C-style array conversion tool will be discussed.
The C-style array tool in particular led to many discussions regarding language semantics, as the transformation is non-trivial when all potential scenarios must be considered.
Furthermore, it is crucial for an automatic refactoring tool to handle all scenarios correctly, as introducing a bug can be difficult to spot across a large codebase and defeats the purpose of automatic refactoring.

\subsection{Array conversions for constant types} \label{subsec:095:030:array_conversions}

The conversion from a C-style \cppinline{int[5]} array to a \cppinline{std::array} is straightforward. The type and size are moved inside the template parameter list, e.g., \cppinline{int array[5]} becomes \cppinline{std::array<int, 5> array}.

A more interesting scenario to consider is how to handle a C-style array of constant integers\\
(\cppinline{const int array[5]}). For this conversion, there are two possibilities. Either with a constant \cppinline{std::array} declaration
(\cppinline{const std::array<int, 5> array}), or with a constant templated type\\
(\cppinline{std::array<const int, 5> array}).
The two solutions are almost identical in meaning, as they will both prevent the programmer from changing the elements inside the \cppinline{std::array}.
The difference between the two approaches is that the non-const member functions\footnote{
    An example of a non-const qualified member function for \cppinline{std::array} is \cppinline{swap} that swaps the contents of two arrays.
} are unavailable for \cppinline{const std::array<..>} version, as it follows the normal rules for const-qualified member functions.
They are technically available to the other version, but calling them will result in a compile-time error as the functions will attempt to modify the values of the \cppinline{const int}s.
As a result, the main distinction for the programmer lies in the type of compile-time error that occurs and the readability of the code.\\
For readability of the code and the error messages, the group prefers the \cppinline{const std::array<..>} representation.
However, in the implementation of the conversion tool, the \cppinline{std::array<const ...>} version is created. This choice was made because it was easier to extract the entire array element type, including the const qualifier than it was to separate the two.
Since the difference between the two versions is so minimal the ease of development was prioritised.

\subsubsection*{Pointer type arrays}
Similar considerations have to be made regarding pointers but they are perhaps more interesting, as they can have multiple const qualifiers.
They can either be pointers to constant objects, constant pointers to objects or constant pointers to constant objects (respectively \cppinline{const int*}, \cppinline{int* const} and \cppinline{const int* const}) \cite{cppreferencePointerDeclarationCppreference}.
Like the array of value types, arrays of pointer types can be converted directly to \cppinline{std::array}s by moving the type into the template parameter list.
E.g., \cppinline{const int*[4]} becomes \cppinline{std::array<const int*, 4>} and \cppinline{int* const[4]} becomes \cppinline{std::array<int* const, 4>}.

It is also possible to represent a constant array with a pointer type through \cppinline{const std::array<int*, N>}.
However, this representation is rather tricky, as it is unclear which of the aforementioned pointer type arrays it corresponds to.
A simple way of testing the similarity between the representations is seen in \cref{code:095:discussion:test_of_array_conversion_eq}.
The idea behind the test is that a \cppinline{const Test*} and \cppinline{const Test* const} should not be allowed to execute the \cppinline{test} method as it is a non-const member function. However, \cppinline{Test* const} should.
This logic can be used to infer, that if \cppinline{a0->test()} compiles, then \cppinline{const std::array<Test*, 1> a0} must be similar to \cppinline{Test* const a0[1]}.
The results indicated that the code compiles successfully.
This means that \cppinline{const std::array<int*, 1>} and \cppinline{std::array<int* const, 1>} are similar except for the differences discussed earlier for the value type arrays. The implementation in the tool uses the \cppinline{std::array<int* const, 1>} version as it is easier to extract from the AST nodes.


\begin{listing}[H]
    \begin{cppcode}
struct Test {void test(){}};
int main() {
    Test t;
    const std::array<Test*, 1> a0 {&t};
    a0[0]->test();
}
    \end{cppcode}
    \caption{Test of conversion similarity
    .}
    \label{code:095:discussion:test_of_array_conversion_eq}
\end{listing}


\subsection{Array parameter conversions}\label{subsec:095disc:arrayParm}
In \cref{sec:080:035:cstyleArrayParm}, the C-style array converter was augmented to transform functions taking constant sized C-style arrays as parameters into functions that accept \cppinline{std::array}s.
While the initial transformation from C-style to \cppinline{std::array} parameters may appear reasonable, further examination reveals that it is a more intricate process.
These intricacies arise due to the diverse ways in which C-style array parameters can be used within the language.
An incomplete list of ways in which C-style arrays can be used in the C++ language is seen below.
Much of this is related to pointer decay (\cref{subsec:085tool:ast_matching}).
\vspace*{-0.75em}
\begin{enumerate}
    \item The size of the C-style array parameter is not enforced, meaning that differently sized C-style arrays can be passed to the function\footnote{
        Passing smaller-sized C-style arrays will cause out-of-bounds errors, assuming the function accesses the specified amount of elements in the array.
    }.
    \item C-style array parameters can be incomplete, i.e., the size can be unspecified.
    \item C-style arrays can be passed to functions expecting pointers as parameters.
\end{enumerate}
\vspace*{-0.75em}
A tool that correctly transforms C-style array parameters would need to take these edge cases into account, which the implementation from \cref{sec:080:035:cstyleArrayParm} does not.
Consequently, using the tool may result in an invalid C++ specification which will result in compilation errors.\\
It is important to note that many of the possible usages of C-style array parameters are discouraged in modern C++ due to their inherent insecurity. 
These patterns often align with the motivations behind utilizing \cppinline{std::array} instead.

An alternative approach could have been to limit the scope of the tool's objectives.
Rather than attempting to refactor the functions themselves, the tool could have opted to pass the \cppinline{std::array}s by their underlying pointers using the \cppinline{data()} method. This approach would pass the underlying, unsafe, pointer to the functions as before, thereby reintroducing the security concerns raised earlier.
The transformation is, however, guaranteed to output a valid specification as many of the edge cases can be ignored.
It would then be possible to develop other tools that implement the transformation of function parameters from C-style arrays to \cppinline{std::array}s.

For instance, one tool could check if all the calls to a given function, which takes a C-style array parameter, are made with \cppinline{data()} methods from \cppinline{std::array}s with the same size. In this scenario, the function parameter could be directly transformed, similar to the current version of the tool. 
Another tool could address cases where the function accepts an incomplete array as a parameter alongside an integral type for bounds-checking purposes (e.g., \cppinline{void func(int arr[], size_t N)}).
If the value of N can be determined at compile-time for all function invocations, the tool could transform the function into a version that accepts a constantly sized array with the size specified as a template parameter (e.g., \cppinline{template<size_t N> void func(std::array<int, N>& arr)}).

Combining multiple tools in such a modular way could be more desirable, as each tool can focus on a subset of the edge cases. This approach opens up the possibility of developing more specialized tools that leverage the full potential of \cppinline{std::array}.
Furthermore, it makes each tool simpler while simultaneously making it possible to modify and upgrade the tool in the future.
