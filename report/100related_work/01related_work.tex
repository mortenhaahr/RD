\chapter{Related work}


The focus of this project has been on deterministic source-code generation using user-provided source-code as the specification. This is, however, not the only methodology of source-code generation.

One example of a different approach to source-code generation comes from microcontroller manufacturers like STMicroelectronics \cite{stmSTMicroelectronicsOurTechnology}. These manufacturers have created tools like STM32CubeIDE which allows the programmer to add startup and configuration code to the code base through a check-box tool \cite{stmSTM32CubeDevelopmentSoftware}. The configuration code is pre-written by the manufacturers and it is used to specify the state of the peripherals inside the microcontrollers.\\
This type of source-code generation is also deterministic, but it uses a pre-defined set of possible combinations as the specification instead of the users' code. This limited specification set provides a higher level of abstraction over the generated configuration and can also serve as documentation for the generated configuration. 


Another approach to automated computing and source-code generation is using probability. 
This type of source-code generation is often employed to generate code from natural language (NL) descriptions of problems, as deterministic models are impractical. 
These probabilistic models allow the generation of programs from descriptions in more familiar languages like English \cite{alonStructuralLanguageModels2020}.
Research in probabilistic automatic programming has been ongoing for many years and the newest tools in the field are actively being used by the public \cite{WhatChatGPTWhy, johnmaedaChoosingLLMModel2023}.
This type of model usually consists of some kind of neural network which is used to process the NL description. 
The result of the processing can then, in some cases, be used as a specification for a deterministic model as shown in \cite{yinSyntacticNeuralModel2017}. 
A probabilistic model can be very useful for source-code generation, even though it is non-deterministic \cite{chenEvaluatingLargeLanguage2021}.