\chapter{Introduction}

The general-purpose programming language C++ has existed since 1985 and is one of the most used programming languages in the world \cite{nexttechnologyprofessionalsMostPopularProgramming2022}.
C++ is known for providing a high level of abstraction without sacrificing performance.
It grants the developer full access to the underlying memory of the system, which makes it a popular choice for systems- and embedded programming  \cite{adminWhereUsedIndustry2021}.
Large-scale projects with millions of code lines, such as LLVM and Folly, have been created in C++ \cite{llvmLLVMCompilerInfrastructure, facebookFollyFacebookOpensource2023}.
These huge projects have to be maintained and updated over time in order to ensure their relevancy and to provide better safety for their users.
Maintaining projects of these magnitudes is a gigantic endeavour, that requires extensive effort in development, code reviews, etc.
By leveraging the usage of automatic tools, developers can streamline various tasks, leading to increased efficiency and reduced manual effort in project upkeep.

The open-source project Clang is a widely-used compiler frontend for languages in the C family, including C++, designed with a modular architecture \cite{clangClangLanguageFamily}.
It encompasses several libraries that offer reusable components, enabling developers to create custom extensions to the project.
One such library is LibTooling, which leverages Clang's lexical- and syntax analysis capabilities to provide access to the abstract syntax tree (AST) of the provided source-code.
With access to the AST, developers can perform semantic analysis and apply sophisticated logic that goes beyond the scope of the compiler.
This logic can then be encapsulated in either standalone tools or plugins that seamlessly integrate with the compiler.

% Purpose of R&D
\begin{quote}
    The purpose of this research and development (R\&D) project is to explore the capabilities of LibTooling, regarding its application in C++ source-code generation.
\end{quote}

When developing source-code generation tools, various semantic considerations need to be taken into account.
These considerations involve making decisions about how to correctly analyze and modify source-code while preserving the intended semantics.\\
This report will not only document the development process but also highlight and discuss some of the significant semantic decisions that arise during tool development.
By exploring these decisions, the report aims to provide valuable insights into the challenges and considerations involved in developing source-code generation tools using LibTooling.\\
To achieve this, three different tools will be created with the LibTooling library.
These tools will serve as a learning platform and provide insights into how Clang represents C++ source-code through its AST. The tools will also explore how information can be extracted from the AST nodes.\\
The three tools which will be developed during this project are:
\vspace*{-0.75em}
\begin{enumerate}
    \item A function renaming tool.
    \item A C-style array\footnote{A C-style array is the classic way of writing arrays in the C programming language, e.g., \cppinline{int array[10]}.} to \cppinline{std::array} converter tool.
    \item A ``to\_string'' function generator tool for enums.
\end{enumerate}