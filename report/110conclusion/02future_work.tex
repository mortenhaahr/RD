\section{Future work}

\subsection*{General tooling}
This project primarily focused on general source-code generation tools that can be applied to diverse codebases.
However, exploring the potential of utilizing LibTooling for tools with more domain-specific or industry-specific use cases would be interesting for further investigation.
A proposed example was made by a Danish software consultancy company that wished to automatically generate ``boilerplate''\footnote{
    In this context, ``boilerplate'' refers to code segments that are frequently used across projects with slight differences.
} code for a frequently employed graphics framework.


\subsection*{C-style array tool}
As discussed in \cref{subsec:095disc:arrayParm}, there are potential benefits associated with altering the implementation of the C-style array tool to avoid directly modifying function declarations.
The proposed tool structure presents an interesting topic for future investigation.

\subsection*{Enum-to-string tool}
In order to further explore different tool structures, the third conceived implementation strategy for the enum-to-string tool (\cref{code:085tool:pseudo_enum3}) could be implemented.
An assessment of the advantages and disadvantages associated with using such a tool structure could then be conducted.

An important edge case was revealed after the project's development phase, which involves the situation where an existing ``to\_string'' function is implemented with a separate definition and declaration.
In such cases, the tool inadvertently updates the declaration with a definition, resulting in the potential creation of multiple definitions, which would lead to compilation errors.
Handling this is nontrivial since the function definition might be located in a separate compilation unit that is inaccessible during the current run of the tool.
To address this, the solution should ensure that declarations are never overwritten.
The tool's new behaviour for each enum declaration can then be summarized as follows:
\vspace*{-0.75em}
\begin{itemize}
    \item If a definition is found, update it.
    \item If a declaration without a corresponding definition is encountered, skip that particular case, as a definition must exist in another compilation unit.
    \item If no declaration or definition exists, generate a new ``to\_string'' function.
\end{itemize}

Addressing this issue further and providing a concrete solution could be interesting to explore in the future.
