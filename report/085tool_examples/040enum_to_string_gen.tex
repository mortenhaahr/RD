\pagebreak
\section{Enum-to-string tools overview}
This section describes a tool that is capable of generating \cppinline{std::string_view to_string(EnumType e)} functions for each enum declaration defined in a C++ program. The \cppinline{to_string} functions take an instance of the enum as an argument and returns a string corresponding to the name of the enumerator.

An example of the outputs of running the tool can be seen in \cref{code:085tool:example_enum}. In the example, in part (1), the enum \cppinline{Animal} is declared with two enumerators: Dog and Cat. In part (2), the \cppinline{to_string} function that would be generated by the tool can be seen. Parts (3) and (4) show another enum declaration with another generated \cppinline{to_string} function.

\begin{listing}[H]
    \begin{cppcode}
// (1): Example enum declaration:
enum class Animal{
    Dog, // Dog is an example of an enumerator (aka. enum constant)
    Cat // Cat is another example of an enumerator
};

// (2): Function that the tool generates:
constexpr std::string_view to_string(Animal e){
    switch(e) {
        case Animal::Dog: return "Dog";
        case Animal::Cat: return "Cat";
    }
}

// (3): Another enum declaration:
enum Greetings {
    ... // enumerators for Greetings
};

// (4): The other enum declaration also gets a to_string function
constexpr std::string_view to_string(Greetings e){
    switch (e) {
        ... 
    }
}
    \end{cppcode}
    \caption{Example (1) declaring an enum in C++ and (2) the \cppinline{to_string} function that the tool generates. In (3) another enum was declared from which another \cppinline{to_string} function is generated (4).}
    \label{code:085tool:example_enum}
\end{listing}

\subsection{Difference from previous tools}

The tool differentiates itself from the previous examples by being a generative tool, meaning that it inserts source code into a file.
In contrast, the renaming tool and C-style comparison tool were refactoring tools that would overwrite existing code lines.
While the differences may seem subtle, it can be more challenging to design generative tools, as the generated code should be syntactically valid as part of the code context.\\
E.g., with the enum-to-string tool, it is necessary to determine if a function named \cppinline{to_string} with the same signature exists in the namespace. This has to be considered since the redefinition of functions is not allowed in C++.
Because such a function can exist, a strategy for handling it must be determined. This could be solved in different ways e.g., by leaving the function untouched or overwriting it. The process of identifying the different edge cases which have to be handled can be very challenging for the tool writers. The only conceivable way of catching the edge cases is by writing tests and running the tool on existing databases. 
It can also be quite a challenge to ensure that all the edge cases which have been identified are handled in the tool. E.g. in the enum-to-string tool, it is quite a complex task to determine if there is an existing ``to\_string'' method as one must analyze the compilation unit for its existence.

Likewise, there are typically extra semantic considerations to be made when designing a generative tool. E.g., if \cppinline{std::string_view to_string(Animal e)} function exists in a namespace ``A'' and \cppinline{Animal} was declared in namespace ``B'', then it would be syntactically correct to add the \cppinline{to_string} function to namespace ``B''. The question of whether it semantically makes sense for these functions to coexist arises
\footnote{This must be determined on a case-by-case basis. E.g. it might make sense for two \cppinline{print(X)} functions to exist in separate namespaces. One that is part of the public API and one that is intended for debugging. However, it might not make sense for two \cppinline{release(X)} functions to exist in separate namespaces as this would indicate there are several ways of releasing the resources allocated in X. (And yet in other cases, it might make perfect sense for two \cppinline{release(X)} functions to exist.)}
and one needs to select a strategy for handling such scenarios.\\
Examples of such strategies could be to ignore the cases, warn the user about them, delete the non-generated version or overwrite the non-generated version.\\
It can be difficult to find and consider all the possible semantic strategies when developing a tool. For some problems, there could be infinite ways to generate the wanted behaviour, like there is when creating a program through a programming language. Some of the possibilities may be better than others but there is still a large design space that could be explored.
To demonstrate this, a list of scenarios where one might need to consider the behaviour of the enum-to-string tool can be seen below. The examples in the list increasingly become more abstract and difficult to implement.

\begin{itemize}
    \vspace{-0.75em}
    \item A \cppinline{to_string} function taking multiple arguments already exists.
    \item The enum is declared privately inside a class.
    \item The enum is declared inside an anonymous namespace.
    \item The enum is declared inside a namespace that by convention is intended to be ignored by users (e.g., \cppinline{detail}, \cppinline{implementation}, etc.).
    \item A function that implements the same behaviour as the generated \cppinline{to_string} function exists.
    \item A function that implements a similar behaviour as the generated \cppinline{to_string} function exists.
    \item A similarly named function exists that implements a similar behaviour as the generated \cppinline{to_string} function exists (e.g. \cppinline{toString}).
    \item ...
\end{itemize}

For the enum-to-string tool, it was decided to overwrite syntactically conflicting implementations of the \cppinline{to_string()} functions.
This has the benefit of allowing the user to change the enum and re-run the tool to see the updated changes. E.g., in \cref{code:085tool:example_enum} if a \cppinline{Animal::Horse} was added to the enum declaration, re-running the tool would update the corresponding \cppinline{std::string_view to_string(Animal e)} function.
However, it also has the downside of essentially reserving the \cppinline{to_string} name leaving the user unable to write their own versions of the function.\\
Furthermore, for this tool, it was decided that if a \cppinline{to_string} function already exists in a different namespace, then the existing version must be overwritten. The reason for implementing this semantic rule was mainly because there are some interesting challenges to consider concerning the recursive traversal of the namespaces, which are described in \cref{subsec:085tool:enum_node_matching_sin}.
\pagebreak
\subsection{Implementations}
The enum-to-string tool is more complex than the previously described tools as it consists of matching on multiple independent declarations of the source code simultaneously, i.e., the enum declarations and the existing \cppinline{to_string} function declarations. The added complexity allows for a wider variety of design approaches, which was shown as part of the project, where three different implementations were considered and two were implemented.

The first approach could implement the tool in a single step, similarly to how it was done in \cref{sec:045tool:c_style_array}. Pseudocode for such a tool can be seen in \cref{code:085tool:pseudo_enum1}. The pseudocode iterates over all the enum declarations in the source code and determines if a \cppinline{to_string} already exists. Depending on the outcome, the function is either updated or generated.

\begin{listing}[H]
    \begin{pythoncode}
for enum_decl in source_files:
    to_string_inst := find(to_string(enum_decl))
    if to_string_inst:
        update to_string_inst
    else:
        generate to_string(enum_decl)
    \end{pythoncode}
    \caption{Pseudocode for the enum-to-string tool.}
    \label{code:085tool:pseudo_enum1}
\end{listing}

An alternative implementation of the tool can be seen in \cref{code:085tool:pseudo_enum2}, which follows a multi-step procedure. The first step consists of updating the existing \cppinline{to_string} functions and saving the relevant enum types in a collection (\textinline{parameters}) for later use. The second step consists of finding all the enum declarations and generating the \cppinline{to_string} functions that were not updated in step 1.

\begin{listing}[H]
    \begin{pythoncode}
parameters := []
for existing_enum_to_string in source_files:
    update existing_enum_to_string
    parameters.append(existing_enum_to_string.parameter)

for enum_decl in source_files:
    if not enum_decl in parameters:
        generate to_string(enum_decl)
    \end{pythoncode}
    \caption{Pseudocode for the enum-to-string tool.}
    \label{code:085tool:pseudo_enum2}
\end{listing}

A third way of implementing the tool can be seen in \cref{code:085tool:pseudo_enum3}, which is also a multi-step procedure. The first step consists of identifying the existing \cppinline{to_string} functions. The second step consists of finding all the enum declarations. If the declaration already has a \cppinline{to_string} function, found in step 1, then it is updated. Otherwise, it is generated.

The main difference between the second and the third implementation is regarding semantics. It might be simpler to divide the tool into two distinct phases consisting of a data collection phase and a post-processing phase, where the post-processing performs the actual logic of the tool - in this case, the updating/generation of the ``to\_string'' function. This appears to be a common division of responsibilities and is among others used in a helper library that was written by Bloomberg \cite{bloombergClangmetatoolFrameworkReusing2023}.

\begin{listing}[H]
    \begin{pythoncode}
parameters := []
for existing_enum_to_string in source_files:
    parameters.append(existing_enum_to_string)

for enum_decl in source_files:
    if enum_decl in parameters:
        update parameters[enum_decl]
    else:
        generate to_string(enum_decl)
    \end{pythoncode}
    \caption{Pseudocode for the enum-to-string tool.}
    \label{code:085tool:pseudo_enum3}
\end{listing}

% Maybe todo: I am not entirely happy with the paragraph below. I don't feel like I am getting my point across
The two designs from \cref{code:085tool:pseudo_enum2} and \cref{code:085tool:pseudo_enum3} have the benefit of being more modular than the one in \ref{code:085tool:pseudo_enum1} since they implement the tool behaviour in multiple steps.
Each step in the designs can essentially be considered independent tools which are then chained together. This makes it possible to split the tool development across team members and also makes it easier to test the tool and reuse it in other projects. For instance in \cref{code:085tool:pseudo_enum2}, the first for-loop can be considered a tool that identifies existing enum-to-string functions, logs them and updates them, and the second for-loop can be considered a tool that generates enum-to-string functions if they are not in the log.

During the project, the tool was implemented with the designs seen in \cref{code:085tool:pseudo_enum1} and \cref{code:085tool:pseudo_enum2}. The design from \cref{code:085tool:pseudo_enum3} was considered but was not implemented. \\The following sections describe the two ``enum-to-string'' tools that were implemented.