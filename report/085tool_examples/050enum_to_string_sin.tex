\section{Enum-to-string -- single step}
The following section describes the enum-to-string tool where the entire tool is implemented as a single \cppinline{ClangTool}.

\subsection{Command line parsing}
Most of the command line parsing was done in a similar manner as in \cref{sec:085tool:example:simple_rename}. However, the behaviour was extended with the introduction of some new command line options for the user to specify.
The command-line option ``in\_place'' was introduced, allowing the user to have the filechanges be written to the terminal instead of directly to the file.
The other option ``debug\_info'' makes the tool print extra debug information to the console.

The additional options were easily introduced as booleans options through the LLVM command line API, as seen in \cref{code:085tool:sin_cl_options}. One simply needs to specify a description and add it to the \cppinline{OptionCategory} -- which is \cppinline{MyToolCategory} in this case.

\begin{listing}[H]
    \begin{cppcode}
static llvm::cl::opt<bool> Inplace(
    "in_place",
    llvm::cl::desc("Inplace edit <file>s, if specified. If not specified the "
                   "generated code will be printed to cout."),
    llvm::cl::cat(MyToolCategory));
static llvm::cl::opt<bool> DebugMsgs(
    "debug_info", llvm::cl::desc("Print debug information to cout."),
    llvm::cl::cat(MyToolCategory));
    \end{cppcode}
    \caption{Implementation of the newly introduced command line options.}
    \label{code:085tool:sin_cl_options}
\end{listing}

The options can then be used as normal booleans throughout the implementation, as seen in \cref{code:085tool:sin_cl_in_place}.

\begin{listing}[H]
    \begin{cppcode}
if (!Inplace) {
    llvm::outs() << new_code.get();
}
    \end{cppcode}
    \caption{Using the \cppinline{Inplace} command line option to print the changes to the command line if \textinline{--in_place} was not specified when running the tool.}
    \label{code:085tool:sin_cl_in_place}
\end{listing}

\subsection{AST node matching}\label{subsec:085tool:enum_node_matching_sin}
% recursively traversing namespaces

\subsection{Node data processing}

\subsection{Handling the results}